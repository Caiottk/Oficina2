\chapter{O objeto 3D digitalizado}\label{cap:ojogo}

A ideia principal do projeto Scan 3D é criar uma solução que aproxime a qualidade e a precisão de uma modelagem tridimensional tradicional, porém, com significativas vantagens em termos de tempo, eficiência e praticidade. Em vez de depender do processo manual e detalhado de modelar objetos do zero utilizando software de design, o Scan 3D digitaliza os objetos físicos diretamente, gerando modelos digitais precisos em um período muito menor. Esta abordagem não só reduz o tempo necessário para a criação dos modelos, mas também simplifica o processo, tornando-o mais acessível e eficiente para diversos usos e aplicações.

\section{Modelagem 3D}

A modelagem 3D surgiu na década de 1960, com o desenvolvimento dos primeiros sistemas de CAD (Computer-Aided Design), que permitiam a criação de modelos digitais de objetos. Inicialmente, essas ferramentas eram utilizadas principalmente na engenharia e na arquitetura para desenhar e simular projetos antes de sua construção. Nos anos 1980, a introdução de computadores pessoais mais poderosos e a redução dos custos de hardware tornaram o CAD mais acessível a um público mais amplo, incluindo designers e artistas gráficos. Nessa época, o desenvolvimento de algoritmos e técnicas de renderização 3D evoluiu rapidamente, permitindo a criação de imagens tridimensionais mais realistas. Softwares como AutoCAD, Blender, e 3D Studio Max começaram a emergir, oferecendo ferramentas robustas para modelagem, animação e renderização 3D.

Com o advento da impressão 3D nos anos 2000, a modelagem 3D ganhou uma nova dimensão. As impressoras 3D permitiram transformar modelos digitais em objetos físicos, revolucionando diversas indústrias, desde a manufatura até a medicina, passando pelo entretenimento e pela moda. O processo de modelagem 3D para impressão começa com a concepção da ideia e o planejamento do modelo, definindo especificações como dimensões, formas, texturas e funcionalidade. Ferramentas de software como AutoCAD, Blender, SolidWorks e Tinkercad são amplamente utilizadas, cada uma com suas próprias características e funcionalidades.

As técnicas de modelagem incluem a modelagem poligonal, que usa polígonos para criar a superfície do objeto e é comum em animação e design de jogos; a modelagem NURBS (Non-Uniform Rational B-Splines), que utiliza curvas matemáticas para definir formas suaves e complexas, ideal para design automotivo e industrial; e a modelagem de escultura digital, que permite modelar objetos como se estivesse esculpindo digitalmente com argila, com softwares como ZBrush sendo populares para esta técnica. Após criar a forma básica do modelo, são aplicadas texturas e materiais para dar ao objeto uma aparência realista, definindo cores, padrões de superfície, reflexividade e outras propriedades visuais.

Antes de imprimir, é crucial verificar se o modelo é adequado para impressão 3D, garantindo que todas as partes do modelo são fechadas, a espessura mínima das paredes e a remoção de qualquer geometria desnecessária. O modelo 3D é então exportado para um formato compatível com impressoras 3D, geralmente STL (Stereolithography) ou OBJ, que contém a informação necessária para que a impressora 3D construa o objeto camada por camada. Finalmente, o arquivo do modelo é carregado em uma impressora 3D, onde o objeto é fisicamente criado usando materiais como plástico, resina, metal ou cerâmica. A impressão pode levar de algumas horas a vários dias, dependendo da complexidade e tamanho do objeto.

A modelagem 3D para impressão 3D é um processo técnico e criativo que envolve várias etapas e ferramentas. Desde sua origem nos sistemas CAD até as modernas técnicas de impressão 3D, a evolução dessa tecnologia tem permitido a criação de objetos cada vez mais complexos e detalhados, transformando a maneira como projetamos e fabricamos produtos no mundo moderno.

\section{O Scan 3D}

A modelagem 3D através do escaneamento de objetos representa uma evolução significativa na maneira como criamos modelos digitais para impressão 3D. Ao invés de começar do zero com software de modelagem, onde cada detalhe precisa ser meticulosamente desenhado, o processo de escaneamento nos permite capturar objetos físicos do mundo real em sua forma tridimensional. Esse avanço não apenas simplifica o processo de criação, mas também aumenta a precisão e a fidelidade do modelo digital ao objeto original.

O escaneamento 3D utiliza tecnologias como sensores infravermelhos e sistemas de captura de imagem para mapear as coordenadas x, y e z do objeto. Com o uso de um kit de desenvolvimento ESP32 e dois motores de passo, a estrutura do scanner permite que o sensor infravermelho seja elevado e percorra o comprimento do objeto, enquanto uma base giratória posiciona o objeto para captura em todos os ângulos. Esse sensor IR montado na estrutura vertical mede as dimensões do objeto em tempo real. Os dados capturados são então processados pelo software Matlab para criar uma representação digital detalhada e precisa do objeto escaneado.

Comparado à modelagem tradicional, o escaneamento 3D oferece várias vantagens. Ele reduz significativamente o tempo necessário para criar modelos 3D complexos, eliminando a necessidade de modelar cada detalhe manualmente. Além disso, permite capturar objetos com formas intricadas e texturas detalhadas que podem ser difíceis de reproduzir através da modelagem digital convencional.

A integração do escaneamento 3D na modelagem para impressão 3D abre novas possibilidades criativas e técnicas. Agora, podemos transformar objetos físicos em modelos digitais com facilidade e precisão, adaptando-os para aplicações que vão desde o design de produtos até a preservação do patrimônio cultural. Essa abordagem inovadora não só simplifica o processo de criação, mas também promove avanços significativos na maneira como exploramos e utilizamos a tecnologia de impressão 3D em diversos setores.

Além das vantagens mencionadas, é importante destacar que nossa proposta visa oferecer um escaneamento 3D de baixo custo, focado em objetos simples, utilizando principalmente um sensor de distância infravermelho. Para projetos que demandem maior complexidade e precisão, como objetos com detalhes finos ou texturas delicadas, seria necessário investir em sensores mais avançados e adotar técnicas de processamento de dados mais sofisticadas. Isso garantiria que o objeto seja digitalizado de forma mais completa e detalhada, atendendo às exigências de aplicações específicas que requerem alta fidelidade no modelo digital.

O uso de tecnologias acessíveis como o sensor infravermelho torna o escaneamento 3D mais acessível e democrático, permitindo que pequenas empresas, artistas e entusiastas de tecnologia tenham acesso a ferramentas poderosas para criação e prototipagem. Com o avanço contínuo da tecnologia, podemos antever um futuro onde o escaneamento 3D de baixo custo se torne ainda mais preciso e versátil, impulsionando inovações em diversas áreas como design de produtos, arquitetura, medicina e muito mais.