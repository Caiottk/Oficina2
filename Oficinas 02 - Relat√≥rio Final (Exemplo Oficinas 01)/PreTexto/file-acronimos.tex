%%%% LISTA DE ABREVIATURAS, SIGLAS E ACRÔNIMOS
%%
%% Relação, em ordem alfabética, das abreviaturas (representação de uma palavra por meio de alguma(s) de sua(s) sílaba(s) ou
%% letra(s)), siglas (conjunto de letras iniciais dos vocábulos e/ou números que representa um determinado nome) e acrônimos
%% (conjunto de letras iniciais dos vocábulos e/ou números que representa um determinado nome, formando uma palavra pronunciável).
%%

%% Este arquivo para definição de abreviaturas, siglas e acrônimos é utilizado com a opção \incluirlistadeacronimos{file} 
%%
%% Vantagens do modo com "file" em relação ao modo "glossaries":
%% 	1) Inserção tabular da listas
%% 	2) Controle da ordem de apresentação das listas
%% 	3) Não é preciso referenciar no texto

\listadeabrevsiglaseacr%% Formatação da lista de abreviaturas, siglas e acrônimos

\begin{listadeabreviaturas}%% Ambiente listadeabreviaturas (Abreviaturas: Representação & Definição \\)
art. & Artigo   \\
cap. & Capítulo \\
sec. & Seção    \\
\end{listadeabreviaturas}

\begin{listadesiglas}%% Ambiente listadesiglas (Siglas: Representação & Definição \\)
ABNT  & Associação Brasileira de Normas Técnicas                                   \\
CNPq  & Conselho Nacional de Desenvolvimento Científico e Tecnológico              \\
EPS   & \textit{Encapsulated PostScript}                                           \\
PDF   & Formato de Documento Portátil, do inglês \textit{Portable Document Format} \\
PS    & \textit{PostScript}                                                        \\
UTFPR & Universidade Tecnológica Federal do Paraná                                 \\
\end{listadesiglas}

\begin{listadeacronimos}%% Ambiente listadeacronimos (Acrônimos: Representação & Definição \\)
Gimp & Programa de Manipulação de Imagem GNU, do inglês \textit{GNU Image Manipulation Program} \\
\end{listadeacronimos}

%% Obs.: comente ou remova as linhas correspondentes ao ambiente que se deseja remover desta lista.
