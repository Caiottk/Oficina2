%%%% CAPÍTULO 5 - CONCLUSÕES E PERSPECTIVAS
%%
%% Deve finalizar o trabalho com uma resposta às
%% hipóteses especificadas na introdução. O autor deve
%% manifestar seu ponto de vista sobre os resultados
%% obtidos; não se deve incluir neste capítulo novos
%% dados ou equações. A partir da tese, alguns assuntos
%% que foram identificados como importantes para serem
%% explorados poderão ser sugeridos como temas para
%% novas pesquisas.

%% Título e rótulo de capítulo (rótulos não devem conter caracteres especiais, acentuados ou cedilha)
\chapter{Conclusões e Perspectivas}\label{cap:conclusoeseperspectivas}

De maneira geral, o 3D Scan está funcionando bem, cumprindo sua função principal de captar todos os pontos de distância ao longo da altura dos objetos. No entanto, para melhorar a precisão e eliminar os erros de captação de distâncias, seria ideal substituir o sensor atual por um modelo superior. Infelizmente, devido a restrições de tempo e orçamento, essa substituição não é viável no momento.

Em resumo, enquanto a estrutura, mecânica e controle de software do projeto estão operando de maneira satisfatória, a precisão da captação de distâncias permanece um desafio a ser resolvido. A troca do sensor poderia potencialmente solucionar esses problemas, mas as limitações atuais nos obrigam a continuar utilizando o sensor disponível, aceitando suas imperfeições no processo de escaneamento.