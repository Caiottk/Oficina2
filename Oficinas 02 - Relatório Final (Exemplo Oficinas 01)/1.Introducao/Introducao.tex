%%%% CAPÍTULO 1 - INTRODUÇÃO
%%
%% Deve apresentar uma visão global da pesquisa, 
%% incluindo: breve histórico, importância e
%% justificativa da escolha do tema, delimitações
%% do assunto, formulação de hipóteses e objetivos
%% da pesquisa e estrutura do trabalho.

%% Título e rótulo de capítulo (rótulos não devem conter caracteres especiais, acentuados ou cedilha)
\chapter{Introdução}\label{cap:introducao}

Com o avanço das impressoras 3D e dos softwares de modelagem, é mister o surgimento de soluções que simplifiquem o processo de criação de objetos e peças para impressão. Devido à complexidade envolvida na constituição de modelagens 3D e à crescente necessidade de precisão e eficiência, a demanda por métodos alternativos para criar modelos tridimensionais de objetos físicos tem aumentado significativamente. Essas novas abordagens visam facilitar a transição do mundo físico para o digital, tornando o processo de modelagem mais acessível e menos trabalhoso.
Uma abordagem que ganhou destaque foi o uso de scanners 3D. Esses dispositivos permitem digitalizar objetos reais em alta precisão, eliminando a necessidade de modelagem manual em softwares de design. Ao digitalizar um objeto com um scanner 3D, é possível obter uma réplica virtual detalhada, economizando tempo e reduzindo a margem de erro humana na modelagem. Dessa forma, a tecnologia de escaneamento 3D se tornou uma ferramenta essencial para profissionais e entusiastas que buscam agilidade e precisão na criação de modelos tridimensionais.

\section{Objetivos}
Ao desenvolver um Scanner 3D, nosso objetivo principal era concluir o projeto dentro do prazo pré-estipulado e com um orçamento reduzido. Queríamos demonstrar que é viável escanear objetos simples de forma eficiente, economizando tempo e recursos que seriam gastos na modelagem 3D tradicional.

\subsection{Objetivos Gerais}

Este projeto tem como objetivo final a criação de um objeto 3D digital que ofereça um resultado comparável ao que seria alcançado através da modelagem manual em um software.

\subsection{Objetivos Específicos}

Utilizando um Esp32 como microcontrolador, queremos:

\begin{itemize}
    \item Desenvolver o dispositivo de digitalização 3D
    \item Mapeamento de coordenadas tridimensionais.
    \item Processamento de dados de captura.
    \item Resolução de dados capturados a partir de parâmetros do sistema. 
    \item Usar display Nextion como interface de usuário do sistema.
\end{itemize}
